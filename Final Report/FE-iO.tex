\section{From Functional Encyption to $\iO$}

In this section we will present a way to use Functional Encryption together with puncturable PRFs and symmetric key encryption in order to achieve $\iO$~\cite{iO-FE}. The idea behind this construction is \emph{first} to reduce the problem of constructing $\iO$ to constructing an obfuscation of only the encryption algorithm of a functional encryption scheme. Then, the obfuscation of a program $f$ will be just the secret key $sk_f$. Note that this would be trivial to achieve if we had a function-hiding public-key functional encryption. However, we only know how to construct function-hiding secret key functional encryption and this is the reason we need to obfuscate the encryption algorithm. The \emph{second} step is to create the $\iO$ of the encryption algorithm without using any $\iO$ but only functional encryption. The way to achieve this is by showing how $\FE$ together with $\iO$ of an encryption algorithm for $n$ bits, can lead to $\iO$ of an encryption algorithm for $n+1$ bits.

In more details, if we want to obfuscate the function $f$ we will create another function $f^*$ such that together with an encryption algorithm $\enc$ as defined below.
\begin{figure}[h!]
\centering
\fbox{
\begin{varwidth}{\textwidth}
$f^*(x,sk,b)$:
\begin{enumerate}
\item constant $\mathsf{CT_0}(=\enc_{sk_0}(f))$
\item constant $\mathsf{CT_1}(=\enc_{sk_1}(f))$
\item Return $\dec_{sk}(\mathsf{CT_b})(x)$
\end{enumerate}
\end{varwidth}}
\end{figure}
\begin{figure}[h!]
\centering
\fbox{
\begin{varwidth}{\textwidth}
$\enc^*(x)$:
\begin{enumerate}
\item constant $pk,sk_0$
\item Return $\enc_{pk}(x,sk_0)$
\end{enumerate}
\end{varwidth}}
\end{figure}

Now, our obfuscated program will just consist of the functional key corresponding to $f^*$, together with the obfuscation of $\enc^*$; in other words, $\iO(f)=(sk_{f^*},\iO(\enc^*))$.

The second step now, is to deal with the obfuscation of the encryption algorithm $\enc^*$. We do this by creating a function $f^*_n$ such that $f^*_n(x)= (\enc^*_{n-1}(x0),\enc^*_{n-1}(x1))$. In other words, this function given an input it returns the encryption of its two possible extensions. Now it is enough to return $sk_{f^*_n}$ together with $\iO(\enc_{n-1})$. Continuing this recursion the obfuscation of $f$ will consist of $n$ functional keys  (one for each $f^*_{i}$) together with two trivial encryptions of $0$ and $1$. Now, if we want to evaluate $f$ on an input $x$, we first need to gradually construct $\enc^*_n(x)$ using the first $n-1$ functional keys and then decrypt using the functional key for $f^*_n$.

\subsection{Formal construction}
The formal construction of the $\iO$ will use recursion. We will assume that we are given a public key functional encryption scheme $\FE$, a single message secure encryption scheme $\enc,\dec$ as long as a puncturable PRF $\{f_k\}_k$.

\begin{figure}[h!]
\centering
\fbox{
\begin{varwidth}{\textwidth}
$\iO(i,C)$:
\begin{enumerate}
\item If $i=1$, return $(C(0),C(1))$
\item Else
\begin{enumerate}
\item Generate secret keys $sk_0,sk_1$ of the secret key encryption scheme
\item Create $\mathsf{CT_0}\gets\enc_{sk_0}(C)$
\item Create $\mathsf{CT_1}\gets\enc_{sk_1}(C)$
\item Define $f(x,sk,b)=\dec_{sk}(\mathsf{CT_b})(x)$
\item Generate $(pk,sk_f)\gets\FESetup(f)$
\item Define $E(x)=\Big(\FEEnc_{pk}(x0,sk_0,0),\FEEnc_{pk}(x1,sk_0,0)\Big)$
\item Return $(\iO(i-1,E),sk_f)$
\end{enumerate}
\end{enumerate}
\end{varwidth}}
\end{figure}
Intuitively, the $\iO$ algorithm obfuscates a program with $n$-size input using an obfuscation of a program with $n-1$-size input. Notice that the output of this algorithm will have the form $O=((\ldots(((v_0,v_1),f_1),f_2),\ldots),f_n)$.


Now, the evaluation algorithm will take as input $O$ together with a value $x$ of $n$ bits and will recursively call itself until it computes the encryption of $x$. Finally it will decrypt using the functional key of the initial function $C$. For a pair $P=(a,b)$, define $P_1=a$ and $P_2=b$. Then the formal definition of the evaluation algorithm is the following:
\begin{figure}[h!]
\centering
\fbox{
\begin{varwidth}{\textwidth}
$\eval(i,E=(E_1,E_2),x\in\bit^n)$:
\begin{enumerate}
\item If $i=1$, return $E_{x_1}$
\item Else return $\FEDec_{E_2}\left(\eval(i-1,E_1,x_{[i-1]})_{x_i}\right)$
\end{enumerate}
\end{varwidth}}
\end{figure}

\label{sec:iO-FE}