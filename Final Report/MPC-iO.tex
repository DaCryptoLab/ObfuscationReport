\section{2-round MPC from $\iO$}

In \cite{MPC-iO} $\iO$ is used as a building block to obtain a 2-round MPC protocol in the CRS model. In MPC a group of mutually distrusting parties aims at computing a certain function of their input without revealing their inputs to each other. We can securely compute any function in this sense whenever the ratio of malicious adversaries is below a certain threshold \cite{}. %XXX: Cite Yao and GMW
% Importance of round complexity (in applications) and what we know about the general case
Much of recent research has focused on improving the efficiency of MPC protocols. An obvious way to measure the efficiency of a MPC protocol is its computational complexity. Round complexity has a certain practical importance, such as in the setting of computing on the web \cite{}, % XXX: Missing citation
where a single server coordinates the computation and parties "log in" at different times without coordination.
The optimal round complexity for MPC in the plain model is known to be 5 rounds \cite{katz2004round}. The work in \cite{MPC-iO} achieves the best bound for MPC in the CRS model as one round protocols are not possible in general.

% General overview on the protocol


% One of the results that stem from iO+OWF


\begin{comment}
What to say:
- 
\end{comment}


% two rounds: second round is a compiler



\subsection{Protocol}

\subsubsection{Model and Tools: NIZK, CRS model, CCA-security}

\subsection{Sketch of Security Proof}

