\section{Impossibility of VBB}
\label{sec:VBB-imp}

In this section we formalize VBB obfuscators for Turing Machines and we sketch a proof of the impossibility result in \cite{VBB-imp}. We will also discuss the impossibility of variants of the original VBB definition as well as the impossibility of applications of obfuscations.

\begin{comment}
In this section:
- definition of VBB
- sketch proof
\end{comment}

\begin{mydef}[TM-obfuscator]
	\label{def:VBB-tm}
	A program $\Obf$ is an obfuscation if the following three conditions hold: % Complexity of O?
	\begin{itemize}
		\item (functionality) For every TM $M$, the string $\ObfM$ describes a machine computing the same function as $M$;
		\item (polynomial slowdown)  For every TM $M$, the description length and the running time of $\ObfM$ are polynomially larger than that of $M$.
		\item (VBB property) For every TM $M$ and PPT $A$ there exists a PPT $S$ s.t. 
		$$ A(\ObfM) \indC S^M(1^{|M|}) $$
		where $\indC$ denotes computational indistinguishability\footnote{Note that in the definition in \cite{VBB-imp} the machine $S$ has oracle access to a polynomially bounded version of $M$. For simplicity, in this report we will abuse notation and simply write $S^M$ referring to a poly-time bounded oracle access. }.
	\end{itemize}
\end{mydef}

The definition above is the most general possible definition of TM-obfuscator: no adversary would be able to get effectively anything more from $\ObfM$ than what it could get from oracle access to $M$. We can increasingly weaken this definition by considering an adversary that, by accessing $\ObfM$, cannot effectively compute a witness $w$ s.t. $R(M,w)$ for any relation $R$, compute any function $f(M)$ or decide any predicate $\pi(M)$.


%It is possible to provide similar definitions for obfuscation of circuits \cite{VBB-imp}.

\subsection{TM-obfuscation is impossible}

To prove the impossibility of obfuscation in the strongest sense (as given in the formal definition above) we can prove that a weaker definition is not possible either. In particular we will focus about an adversary that cannot compute effectively any predicate of $M$ when having access to $\ObfM$.
We will give an intuition of the original proof by proving that 2-TM obfuscators do not exist. Adapting the proof for TM-obfuscators is straightforward.

\begin{mydef}[(Predicate) 2-TM obfuscators ]
	A 2-TM obfuscator is defined in the same way as a TM obfuscator, except that the "VBB property" is defined as follows:
	\begin{itemize}
		\item (VBB property) For every pair of TM $M_1, M_2$ and PPT $A$ there exists a PPT $S$ s.t. 
		$$ |\Pr[A(\Obf(M_1), \Obf(M_2)) - \Pr[ S^{M_1,M_2}(1^{|M_1|+|M_2|})]| \leq \negl(min(|M_1|, |M_2|)) $$
	\end{itemize}
\end{mydef} 

% Proof
The main idea behind the proof is that there is a difference between having oracle access to a function and getting a machine that can compute it. In the latter case we have a string on which we can do computations, possibly learning something from it.
\newcommand{\UA}{\function{UA}}
\newcommand{\T}{\function{T}}
We will then consider a function that cannot be learned by oracle access. Consider the Turing Machine
 $\function{UltimateAnswer}_q$, $\UA_q$ for short, that returns the 
 answer to the ultimate question $q \in \bit^k$ on "life, the universe and everything" \cite{adams}.
\[
\UA_q(x) =
\begin{cases} 
\hfill 42    \hfill & \text{ if $x = q$} \\
\hfill 0 \hfill & \text{otherwise} \\
\end{cases}
\]
Now consider another Turing Machine $\function{TestUltimateAnswer}_q$, $\T_q$ for short, which given in input a machine $M$ distinguishes whether it provides the answer to the ultimate question $q \in \bit^k}$ or not:
\[
\T_q(M) =
\begin{cases} 
\hfill 1    \hfill & \text{ if $M(q) = 42$} \\
\hfill 0 \hfill & \text{otherwise} \\
\end{cases}
\]



\subsection{Other impossibility results}

% Circuits
The impossibility results for TM obfuscations also extend to other types of obfuscations.
In \cite{VBB-imp} the authors define \emph{circuit obfuscators}, the circuit counterpart of TM obfuscators. They also providethe following result:

\begin{mytheorem}
	If one-way functions exist, then circuit obfuscators do not exist.
\end{mytheorem}

% Unobfuscatable functions %XXX: Maybe, maybe not
% Approximate obfuscators
As discussed so far we cannot that hope for natural, strong definitions of obfuscation to exist. We will now present results in \cite{VBB-imp} on weaker variants of obfuscations and potential applications.

One of the requirements we made in Definition \ref{def:VBB-tm} was that the obfuscated program should compute \emph{exactly} the same function as the original program. Could we obtain obfuscation if we relaxed our definition by allowing obfuscators to produce a program "approximately" the same as the original one?
The answer is negative. We can define an $\epsilon$-approximate obfuscator as one that returns on all inputs give the correct answer with probability higher than $1-\epsilon$ \cite{VBB-imp}. As for circuit obfuscation, if one-way functions exist, then $\epsilon$-approximate obfuscators do not exist.

% Applications: types of encryption...
We could consider these negative results as the byproduct of small issues in our definition of VBB. In \cite{VBB-imp} the authors show this is not the case and that there is something inherently flawed in the idea of "Virtual Black Box". Maybe, despite these general impossibility results --- they suggest --- we can still obtain obfuscation for specific applications. Unfortunately this is not the case. They consider applications such as: signature schemes, private-key encryption, pseudorandom functions and message authentication codes. They prove that, if each of these primitives exists, then some instances of them must be "unobfuscatable".
Finally they show that, conditional to Decisional Diffie Helman being "hard", there exist families of low-complexity functions (i.e. computable by $TC_0$ circuits) that are unobfuscatable.
Later \cite{VBB-imp-aux} presented other impossibility results on VBB obfuscation proving its impossibility for adversaries with auxiliary inputs.

% Watermarking
Another consequence of the results in \cite{VBB-imp} is the impossibility of general \emph{(fragile) watermarking} algorithms \cite{collberg2002watermarking}. 
Watermarking is the problem of marking an object in a way that it would be hard to remove the mark without inevitably ruining the object. \cite{VBB-imp} presents two definitions of watermarking and prove their (conditional) impossibility, building on the previous results on VBB obfuscation.


% Definition here