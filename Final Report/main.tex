\documentclass[]{article}

\usepackage{comment}
\usepackage{amsfonts,amsmath,amsthm}
\usepackage[noend]{algpseudocode}
\usepackage{varwidth}
\newcommand{\function}[1]{\ensuremath{\mathsf{#1}}}

\newtheorem{mydef}{Definition}
\newtheorem{mytheorem}{Theorem}
\newtheorem{myinformaltheorem}{Informal Theorem}

\newcommand{\iO}{i\mathcal{O}}
\newcommand{\Obf}{\mathcal{O}}
\newcommand{\ObfM}{\mathcal{O}(M)}

\newcommand{\indC}{\sim_c}
\newcommand{\compind}{\indC}

%oracle indistinguishability
\newcommand{\orind}{\overset{o}{\sim}}

\newcommand{\negl}{\function{neg}}

% VBB proof
\newcommand{\UA}{\function{UA}}
\newcommand{\T}{\function{T}}
\newcommand{\Z}{\function{Z}}

% PK commands
\newcommand{\pkGen}{\function{Gen}}
\newcommand{\pkEnc}{\function{Enc}}
\newcommand{\pkDec}{\function{Dec}}

\newcommand{\Prog}{\function{Prog}}

\newcommand{\cclass}{\mathcal{C}}
\newcommand{\fclass}{\mathcal{F}}

\newcommand{\bit}{\ensuremath{\{0,1\}}}
\newcommand{\binstrings}{\bit^{*}}

%Functional Encryption commands
\newcommand{\FE}{\ensuremath{\mathsf{FE}}}
\newcommand{\FESetup}{\ensuremath{\mathsf{FE.Setup}}}
\newcommand{\FEEnc}{\ensuremath{\mathsf{FE.Enc}}}
\newcommand{\FEDec}{\ensuremath{\mathsf{FE.Dec}}}


%Deniable Encryption commands
\newcommand{\DE}{\ensuremath{\mathsf{DE}}}
\newcommand{\DESetup}{\ensuremath{\mathsf{DE.Setup}}}
\newcommand{\DEEnc}{\ensuremath{\mathsf{DE.Enc}}}
\newcommand{\DEDec}{\ensuremath{\mathsf{DE.Dec}}}
\newcommand{\DEExp}{\ensuremath{\mathsf{DE.Exp}}}


%Puncturable PRFS
\newcommand{\prfgen}{\ensuremath{\mathsf{Gen}}}
\newcommand{\prfpunc}{\ensuremath{\mathsf{Punc}}}

%Symmetric encryption
\newcommand{\senc}{\ensuremath{\mathsf{SEnc}}}
\newcommand{\sdec}{\ensuremath{\mathsf{SDec}}}

%Public key encryption
\newcommand{\pgen}{\ensuremath{\mathsf{PGen}}}
\newcommand{\penc}{\ensuremath{\mathsf{PEnc}}}
\newcommand{\pdec}{\ensuremath{\mathsf{PDec}}}

%General encryption
\newcommand{\gen}{\ensuremath{\mathsf{Gen}}}
\newcommand{\enc}{\ensuremath{\mathsf{Enc}}}
\newcommand{\dec}{\ensuremath{\mathsf{Dec}}}

\newcommand{\explain}{\ensuremath{\mathsf{Explain}}}

%opening
\title{Obfuscation: Impossibility Results and Applications}
\author{Matteo Campanelli \and Marios Georgiou}

\begin{document}

\maketitle

\begin{abstract}
In this report we discuss some major results in the area of obfuscation. In particular we show that VBB obfuscation is impossible, and that a weaker definition of obfuscation, called \emph{indistinguishability obfuscation} (IO) can lead to constructions of several important cryptographic primitives. In short, IO guarantees that the obfuscations of two functionally equivalent programs are indistinguishable. Here we show how we can use IO together with one-way functions to create a secure two-round MPC protocol as well as a deniable public-key encryption scheme. Last, we show how we can use functional encryption in order to create IO.

\end{abstract}


\section{Introduction}
% General intro on obfuscation
The goal of obfuscation is to make a program unintelligible preserving its functionality.
Cryptographers first conceived obfuscation as \emph{Virtual Black Box obfuscation} (VBB) where
no polynomial adversary can gain more information from using the obfuscated program than from having
oracle access to the function it computes. In 2001 \cite{VBB-imp} showed that it is impossible to have a general construction of obfuscators for Turing Machines. These negative results extend to the obfuscation of circuits as well as to some weaker definitions (e.g. approximate obfuscation) and applications (e.g. software watermarking).

After this result, researched focused on weaker, yet meaningful, definitions of
obfuscation. Currently the most promising one is that of \emph{indistinguishability Obfuscation} ($\iO$), suggested in \cite{VBB-imp}. In $\iO$, no PPT can distinguish between two functionally equivalent obfuscated programs. Recently, \cite{garg2013candidate} proposed an efficient construction of $\iO$ for all circuits basing its security on assumptions related to multilinear maps \cite{garg2013candidatemulti}. An alternative construction from Functional Encryption is in \cite{iO-FE}.

Clearly $\iO$ is a weaker primitive than VBB obfuscation. We have little hope of proving that $\iO$ implies one-way functions: if $P = NP$ one-way functions would not exist but we would still have $\iO$ (\cite{VBB-imp} describes a general but inefficient construction of $\iO$). 
% In the next paragraph I say that iO is weak but we can still build interesting things by using it combining it with other assumptions
Thus it is unlikely we will be able to construct "interesting" cryptographic primitives from
$\iO$ alone (one exception is witness encryption \cite{garg2013witness}) 
We can however combine $\iO$ with one-way functions (OWFs) to construct several other primitives \cite{iO-deniable}, such as functional encryption, public-key encryption, NIZK, CCA encryption and deniable encryption.

% XXX: Say more about the results we are going to talk about above here?

In this report we discuss some of the current state of knowledge about what obfuscation can and cannot do.
We first discuss some of the impossibility results on VBB obfuscation (Section \ref{sec:VBB-imp}). 
We then describe some constructions obtained by combining $\iO$ and OWFs. In Section \ref{sec:iO-mpc} we present a two-round MPC protocol in the Common Reference String (CRS) model due to \cite{MPC-iO}. Next, in Section \ref{sec:iO-deniable}, we present a construction of deniable encryption \cite{iO-deniable}. Finally, in Section \ref{sec:iO-FE} we describe how we can construct $\iO$ from subexponentially secure functional encryption 
\cite{iO-FE}.





\section{Preliminaries}

Here we define all cryptographic primitives that we will use in this report. We will say that a function $f(\lambda)$ is negligible if $f(\lambda)\in o(1/p(\lambda))$ for any polynomial $p$. In this case we will abuse notation and write $f(\lambda)=\negl(\lambda)$.

Two distributions $D_0,D_1$ over a universe $U\in\bit^n$ are called indistinguishable, denoted by $D_0\compind D_1$, if for any PPT algorithm $A$, it holds that
\[
\left|\Pr_{x\gets D_0}[A(x)=1]-\Pr_{x\gets D_1}[A(x)=1]\right|=\negl(n)
\]

Two algorithms $A,B$ are called \emph{oracle indistinguishable}, denoted by $A\orind B$, if for any PPT algorithm $D$, it holds that
\[
\left|\Pr\left[D^A=1]-\Pr[D^B=1\right]\right|=\negl(n)
\]

\subsection{Virtual Black Box Obfuscation}
\label{subsec:VBB}
Virtual black box (VBB) obfuscation is the strongest kind of obfuscation we can hope for. Unfortunately, as we will show, it is impossible to achieve.

\begin{mydef}[TM-obfuscator]
	\label{def:VBB-tm}
	A program $\Obf$ is an obfuscation if the following three conditions hold: % Complexity of O?
	\begin{itemize}
		\item (functionality) For every TM $M$, the string $\ObfM$ describes a machine computing the same function as $M$;
		\item (polynomial slowdown)  For every TM $M$, the description length and the running time of $\ObfM$ are polynomially larger than that of $M$.
		\item (VBB property) For every TM $M$ and PPT $A$ there exists a PPT $S$ s.t. 
		$$ A(\ObfM) \indC S^M(1^{|M|}) $$
		where $\indC$ denotes computational indistinguishability\footnote{Note that in the definition in \cite{VBB-imp} the machine $S$ has oracle access to a polynomially bounded version of $M$. For simplicity, in this report we will abuse notation and simply write $S^M$ referring to a poly-time bounded oracle access. }.
	\end{itemize}
\end{mydef}

The definition above is the most general possible definition of TM-obfuscator: no adversary would be able to get effectively anything more from $\ObfM$ than what it could get from oracle access to $M$. We can increasingly weaken this definition by considering an adversary that, by accessing $\ObfM$, cannot effectively compute a witness $w$ s.t. $R(M,w)$ for any relation $R$, compute any function $f(M)$ or decide any predicate $\pi(M)$.



\subsection{Indistinguishability Obfuscation}

Intuitively an indistinguishability obfuscator $\iO$ is an algorithm that takes as input a circuit and outputs an obfuscated circuit that preserves the functionality of the original one. The security definition states that the indistinguishability obfuscations of two functionally equivalent programs are indistinguishable.

\begin{mydef}[IO]
A PPT algorithm $\iO$ is an indistinguishability obfuscator for a class of circuits $\cclass$ if it has the following properties:
\begin{itemize}
\item (functionality) For any $C\in\cclass$ and for any input $x$, it holds that
\[
\iO(C)(x)=C(x)
\]
\item (indistinguishability) For any functionally equivalent circuits $C_0,C_1\in\cclass$, it holds that
\[
\iO(C_0,1^\lambda)\compind \iO(C_1,1^\lambda)
\]
where the distributions are over the randomness of $\iO$.
\end{itemize}
\end{mydef}

\subsection{Tools for building MPC from $\iO$: UC, CRS, NIZK and CCA}

% UC
Universal Composition (UC) \cite{canetti2001universally} is a stringent security framework that guarantees secure under concurrent general composition with arbitrary sets of parties. For more details, the reader is referred to \cite{canetti2001universally}.

% CRS % XXX: Citation
In the common reference string (CRS) model \cite{canetti2001comm,canetti2002}, all parties in the system obtain from a trusted party a reference string, which is sampled according to a pre-specified distribution D. The reference string is referred to as the CRS.


%NIZK
Let $R$ be an efficiently computable binary relation. For pairs $(x, w) \in R$ we call $x$ the statement and $w$ the witness. A non-interactive proof system \cite{blum1988non} for a relation $R$ consists of a common reference string generation algorithm $K$, a prover $P$ and a verifier $V$. The generation algorithm produces a string $\sigma$ common to prover and verifier.
The prover takes as input $(\sigma, x, w)$ and produces a proof $\pi$. The verifier takes as input $(\sigma, x, \pi)$ and outputs 1 if the proof is acceptable and 0 if rejecting the proof.

% CCA
Informally, an encryption scheme is secure against chosen-ciphertext attacks (CCA) if it is secure against an attacker can choose a ciphertext and obtain its decryption under an unknown key. In the attack, an adversary has a chance to enter one or more known ciphertexts into the system and obtain the resulting plaintexts.
For more details on (public-key) CCA-secure schemes the reader can consult \cite{katz2014introduction}.


\subsection{Functional Encryption}
A functional encryption scheme is an encryption scheme where each decryption key is associated to a function. Using this decryption key, we are not able to learn the original message but only the function of this message. In the construction of $\iO$ from $\FE$ we are only interested in a weaker version of $\FE$ where we have only one function $f$ for each public key. Moreover, it is enough assume selective security.
\begin{mydef}[Functional Encryption]
A single key, selectively secure public-key succinct functional encryption scheme $\FE$ for a function class $\fclass$ consists of three PPT algorithms: $\FESetup(f,1^\lambda)\to(pk,sk_f)$, $\FEEnc_{pk}(x)\to c$ and $\FEDec_{sk_f}(c) \to y$ and satisfies the following properties:
\begin{itemize}
\item (correctness) For any $f\in\fclass$, if $(pk,sk_f)\gets\FESetup(f,1^\lambda)$ then
\[
\FEDec_{sk_f}\left(\FEEnc_{pk}(x)\right)=f(x)
\]
\item (selective security): For any $x_0,x_1\in\bit^n$ and $f\in\fclass$ such that $f(x_0)=f(x_1)$, if $(pk,sk_f)\gets\FESetup(f,1^\lambda)$ then
\[
\left(pk,sk_f,\FEEnc_{pk}(x_0)\right)\compind \left(pk,sk_f,\FEEnc_{pk}(x_1)\right)
\]
\item Succinct encryption: the size of the encryption circuit is polynomial in $\lambda,n$. In particular, it does not depend on $\fclass$.
\end{itemize}
\end{mydef}


\subsection{Pseudorandomness}

In our constructions we need to make use of some pseudorandom tools, defined in this section. 

\begin{mydef}[PRG]
A deterministic function $G:\bit^n\to\bit^{2n}$ is called a length doubling pseudorandom generator if it is efficiently computable and $G(U_n)\compind U_{2n}$.
\end{mydef}

Another important tool which is used in almost any construction involving $\iO$ is that of \emph{puncturable PRFs}. Intuitively, a puncturable PRF $f$ has the property that we can \emph{puncture out} a point $x^*$ from its domain in such a way that $f$ maintains it functionality on all inputs $x\ne x^*$ but it is pseudorandom on $x^*$. It is worth to note that the actual puncturing takes place on the key. In other words, if we want to puncture out $x^*$, we will create a key $k\{x^*\}$ and now $f_{k\{x^*\}}$ will be the puncturable PRF (the definition of $f$ remains the same.)

\begin{mydef}[Puncturable PRFs]
A family of functions $\{f_k:\bit^n\to\bit^n\}_{k\in\bit^n}$ is called a puncturable PRF if it is efficiently computable and there exist PPT algorithms $\prfgen(1^n)\to k$ and $\prfpunc(x^*)\to k\{x^*\}$ such that
\begin{itemize}
\item (functionality preserving) For any $x\ne x^*$, if $k\gets\prfgen(1^n)$ and $k\{x^*\}\gets\prfpunc(x^*)$ it holds that
\[
f_k(x)=f_{k\{x^*\}}(x)
\]
\item (pseudorandom on the punctured point) For any $x^*$, if $k\gets\prfgen(1^n)$ and $k\{x^*\}\gets\prfpunc(x^*)$ then
\[
\left(x^*,k\{x^*\},f_k(x^*)\right)\compind \left(x^*,k\{x^*\},U\right)
\]
where the distributions are over the randomness of $\prfgen$ and $\prfpunc$.
\end{itemize}
\end{mydef}

In the following we will also make use of a secret-key encryption scheme $(\senc,\sdec)$ which is single message secure; i.e. for any two messages $m_0,m_1$ it holds that $\senc_k(m_0)\compind\senc_k(m_1)$, where $k\gets\bit^n$.


\subsection{Deniable Encryption}
Suppose that we have retrieved from a user her public key $pk$ and we have encrypted a message $m$ under $pk$ and retrieved a ciphertext $c$. Later on, someone may force us to reveal to him the original message by revealing the initial randomness we used. If we were able to find a randomness for any possible message then we would be able to avoid this situation. Informally, the scheme is deniable if given $c$ and any possible message $m'$ we are able to find an explanation randomness $e$, such that if we encrypt $m'$ with $e$ we get again the same ciphertext $c$. The scheme is publicly deniable if moreover, we can explain the ciphertext without need of the original randomness.

\begin{mydef}[Deniable public-key encryption]
A publicly deniable encryption scheme consists of the following three algorithms: $\DESetup(1^\lambda)\to(pk,sk)$, $\DEEnc_{pk}(m,u)\to c$ (where $u$ is the randomness), $\DEDec_{sk}(c)\to m$ and $\DEExp(pk,c,m,r)\to e$ (where $r$ is the randomness) and satisfies the following properties:
\begin{itemize}
\item (correctness) For any $m$, if $(pk,sk)\gets \DESetup(1^\lambda)$ then
\[
\DEDec_{sk}\left(\DEEnc_{pk}(m)\right)=m
\]
\item (IND-CPA) For $b\in\bit$ define oracles $C^b(m_0,m_1)=\DEEnc_{pk}(m_b,u)$. Then
\[
(pk,C^0)\orind(pk,C^1)
\]
\item (indistinguishability of explanations) Let $O^0(m)=\left(\DEEnc_{pk}(m,u),u\right)$ and $O^1(m)=\left(\DEEnc_{pk}(m,u),\DEExp\left(pk,\DEEnc_{pk}(m,u),m,r\right)\right)$. Then
\[
O^0\orind O^1
\]
\end{itemize}
\end{mydef}
Intuitively, the indistinguishability of explanations property states that the adversary cannot distinguish if she receives the original randomness or the explanation.


\section{Impossibility of VBB}
\label{sec:VBB-imp}

In this section we discuss the impossibility of VBB obfuscators for Turing Machines \cite{VBB-imp}. We will also discuss the impossibility of variants of the original VBB definition as well as the impossibility of applications of obfuscations.

%It is possible to provide similar definitions for obfuscation of circuits \cite{VBB-imp}.

\subsection{TM-obfuscation is impossible}

To prove the impossibility of obfuscation in the strongest sense (as given in the formal definition above) we can prove that a weaker definition is not possible either. In particular we will focus about an adversary that cannot compute effectively any predicate of $M$ when having access to $\ObfM$ (see discussion at the end of Subsection \ref{subsec:VBB}).
We will give an intuition of the original proof by proving that 2-TM obfuscators do not exist. Adapting the proof for TM-obfuscators is straightforward.

\begin{mydef}[(Predicate) 2-TM obfuscators ]
	A 2-TM obfuscator is defined in the same way as a TM obfuscator, except that the ``VBB property" is defined as follows:
	\begin{itemize}
		\item (VBB property) For every pair of TM $M_1, M_2$ and PPT $A$ there exists a PPT $S$ s.t. 
		$$ |\Pr[A(\Obf(M_1), \Obf(M_2)) - \Pr[ S^{M_1,M_2}(1^{|M_1|+|M_2|})]| \leq \negl(\min(|M_1|, |M_2|)) $$
	\end{itemize}
\end{mydef} 

% Proof
The main idea behind the proof is that there is a difference between having oracle access to a function and getting a machine that can compute it. In the latter case we have a string on which we can do computations, possibly learning something from it.
We will then consider a function that cannot be learned by oracle access. Consider the Turing Machine
 $\function{UltimateAnswer}_q$, $\UA_q$ for short, that returns the 
 answer to the ultimate question $q \in \bit^k$ on ``life, the universe and everything" \cite{adams}.
\[
\UA_q(x) =
\begin{cases} 
\hfill 42    \hfill & \text{ if $x = q$} \\
\hfill 0 \hfill & \text{otherwise} \\
\end{cases}
\]
Now consider another Turing Machine $\function{TestUltimateAnswer}_q$, $\T_q$ for short, which given in input a machine $M$ distinguishes whether it provides the answer to the ultimate question $q \in \bit^k$ or not:
\[
\T_q(M) =
\begin{cases} 
\hfill 1    \hfill & \text{ if $M(q) = 42$} \\
\hfill 0 \hfill & \text{otherwise} \\
\end{cases}
\]

% There is an adversary that can learn "a lot" even from obfuscation.
Consider a very simple adversary that given two (obfuscated) TMs as input simply runs the second TM on the first one, i.e. $A(M_1, M_2) = M_2(M_1)$. Having access to obfuscations of $\UA_q$ and $\T_q$ $A$ can clearly learn something. In particular, for any $q \in \bit^k$:
\begin{equation} \Pr[A(\Obf(\UA_q), \Obf(\T_q) = 1] = 1 \end{equation}

What can we say about what an arbitrary poly-time machine $S$ can learn from $\UA_q$ and $\T_q$ having oracle access to them? $S$ would not be able to pass the representation of $\UA_q$ as $A$ could. Also, it would be extremely unlikely for $S$ to sample the only point $q$ where $\UA_q$ does not return $0$. If we call $\Z$ the identically null function, we have:
\begin{equation} |\Pr[S^{\UA_q, \T_q}(1^k)] - \Pr[S^{\Z, \T_q}(1^k)]| \leq 2^{\Omega(k)}\end{equation}

On the other hand \begin{equation} \Pr[A(\Obf(\Z), \Obf(\T_q) = 1] = 0 \end{equation}
Combining $(1)$, $(2)$ and $(3)$ it is easy to show a contradiction.

\subsection{Other impossibility results}

% Circuits
The impossibility results for TM obfuscations also extend to other types of obfuscations.
In \cite{VBB-imp} the authors define \emph{circuit obfuscators}, the circuit counterpart of TM obfuscators. They also providethe following result:

\begin{mytheorem}
	If one-way functions exist, then circuit obfuscators do not exist.
\end{mytheorem}

% Unobfuscatable functions %XXX: Maybe, maybe not
% Approximate obfuscators
As discussed so far we cannot hope for natural, strong definitions of obfuscation to exist. We will now present results in \cite{VBB-imp} on weaker variants of obfuscations and potential applications.

One of the requirements we made in Definition \ref{def:VBB-tm} was that the obfuscated program should compute \emph{exactly} the same function as the original program. Could we obtain obfuscation if we relaxed our definition by allowing obfuscators to produce a program ``approximately" the same as the original one?
The answer is negative. We can define an $\epsilon$-approximate obfuscator as one that returns on all inputs give the correct answer with probability higher than $1-\epsilon$ \cite{VBB-imp}. As for circuit obfuscation, if one-way functions exist, then $\epsilon$-approximate obfuscators do not exist.

% Applications: types of encryption...
We could consider these negative results as the byproduct of small issues in our definition of VBB. In \cite{VBB-imp} the authors show this is not the case and that there is something inherently flawed in the idea of "Virtual Black Box". Maybe, despite these general impossibility results --- they suggest --- we can still obtain obfuscation for specific applications. Unfortunately this is not the case. They consider applications such as: signature schemes, private-key encryption, pseudorandom functions and message authentication codes. They prove that, if each of these primitives exists, then some instances of them must be ``unobfuscatable".
Finally they show that, conditional to Decisional Diffie Helman being ``hard", there exist families of low-complexity functions (i.e. computable by $TC_0$ circuits) that are unobfuscatable.
Later \cite{VBB-imp-aux} presented other impossibility results on VBB obfuscation proving its impossibility for adversaries with auxiliary inputs.

% Watermarking
Another consequence of the results in \cite{VBB-imp} is the impossibility of general \emph{(fragile) watermarking} algorithms \cite{collberg2002watermarking}. 
Watermarking is the problem of marking an object in a way that it would be hard to remove the mark without inevitably ruining the object. \cite{VBB-imp} presents two definitions of watermarking and prove their (conditional) impossibility, building on the previous results on VBB obfuscation.


% Definition here

\section{2-round MPC from $\iO$}


% Importance of round complexity (in applications) and what we know about the general case

% One of the results that stem from iO+OWF


\begin{comment}
What to say:
- 
\end{comment}

% General overview on the protocol
% two rounds: second round is a compiler



\subsection{Protocol}

\subsubsection{Model and Tools: NIZK, CRS model, CCA-security}

\subsection{Sketch of Security Proof}



\section{From $\iO$ to deniable encryption}

\section{From Functional Encyption to $\iO$}

In this section we will present a way to use Functional Encryption together with puncturable PRFs and symmetric key encryption in order to achieve $\iO$~\cite{iO-FE}. The idea behind this construction is \emph{first} to reduce the problem of constructing $\iO$ to constructing an obfuscation of only the encryption algorithm of a functional encryption scheme. Then, the obfuscation of a program $f$ will be just the secret key $sk_f$. Note that this would be trivial to achieve if we had a function-hiding public-key functional encryption. However, we only know how to construct function-hiding secret key functional encryption and this is the reason we need to obfuscate the encryption algorithm. The \emph{second} step is to create the $\iO$ of the encryption algorithm without using any $\iO$ but only functional encryption. The way to achieve this is by showing how $\FE$ together with $\iO$ of an encryption algorithm for $n$ bits, can lead to $\iO$ of an encryption algorithm for $n+1$ bits.

In more details, if we want to obfuscate the function $f$ we will create another function $f^*$ such that together with an encryption algorithm $\enc$ as defined in Figure \ref{fig:enc}.
\begin{figure}[!h]
	
\centering
\fbox{
\begin{varwidth}{\textwidth}
$f^*(x,sk,b)$:
\begin{enumerate}
\item constant $\mathsf{CT_0}(=\enc_{sk_0}(f))$
\item constant $\mathsf{CT_1}(=\enc_{sk_1}(f))$
\item Return $\dec_{sk}(\mathsf{CT_b})(x)$
\end{enumerate}
\end{varwidth}}
\hfill
\centering
\fbox{
\begin{varwidth}{\textwidth}
$\enc^*(x)$:
\begin{enumerate}
\item constant $pk,sk_0$
\item Return $\enc_{pk}(x,sk_0)$
\end{enumerate}
\end{varwidth}}
\caption{}
\label{fig:enc}
\end{figure}

Now, our obfuscated program will just consist of the functional key corresponding to $f^*$, together with the obfuscation of $\enc^*$; in other words, $\iO(f)=(sk_{f^*},\iO(\enc^*))$.

The second step now, is to deal with the obfuscation of the encryption algorithm $\enc^*$. We do this by creating a function $f^*_n$ such that $f^*_n(x)= (\enc^*_{n-1}(x0),\enc^*_{n-1}(x1))$. In other words, this function given an input it returns the encryption of its two possible extensions. Now it is enough to return $sk_{f^*_n}$ together with $\iO(\enc_{n-1})$. Continuing this recursion the obfuscation of $f$ will consist of $n$ functional keys  (one for each $f^*_{i}$) together with two trivial encryptions of $0$ and $1$. Now, if we want to evaluate $f$ on an input $x$, we first need to gradually construct $\enc^*_n(x)$ using the first $n-1$ functional keys and then decrypt using the functional key for $f^*_n$.

\subsection{Formal construction}
The formal construction of the $\iO$ will use recursion. We will assume that we are given a public key functional encryption scheme $\FE$, a single message secure encryption scheme $\enc,\dec$ as long as a puncturable PRF $\{f_k\}_k$.

\begin{figure}[h!]
\centering
\fbox{
\begin{varwidth}{\textwidth}
$\iO(i,C)$:
\begin{enumerate}
\item If $i=1$, return $(C(0),C(1))$
\item Else
\begin{enumerate}
\item Generate secret keys $sk_0,sk_1$ of the secret key encryption scheme
\item Create $\mathsf{CT_0}\gets\enc_{sk_0}(C)$
\item Create $\mathsf{CT_1}\gets\enc_{sk_1}(C)$
\item Define $f(x,sk,b)=\dec_{sk}(\mathsf{CT_b})(x)$
\item Generate $(pk,sk_f)\gets\FESetup(f)$
\item Define $E(x)=\Big(\FEEnc_{pk}(x0,sk_0,0),\FEEnc_{pk}(x1,sk_0,0)\Big)$
\item Return $(\iO(i-1,E),sk_f)$
\end{enumerate}
\end{enumerate}
\end{varwidth}}

\end{figure}
Intuitively, the $\iO$ algorithm obfuscates a program with $n$-size input using an obfuscation of a program with $n-1$-size input. Notice that the output of this algorithm will have the form $O=((\ldots(((v_0,v_1),sk_{f_1}),sk_{f_2}),\ldots,sk_{f_{n-1}}),sk_{f_n})$.


Now, the evaluation algorithm will take as input $O$ together with a value $x$ of $n$ bits and will recursively call itself until it computes the encryption of $x$. Finally it will decrypt using the functional key of the initial function $C$. For a pair $P=(a,b)$, define $P_1=a$ and $P_2=b$. Then the formal definition of the evaluation algorithm is the following:
\begin{figure}[h!]
\centering
\fbox{
\begin{varwidth}{\textwidth}
$\eval(i,E=(E_1,E_2),x\in\bit^n)$:
\begin{enumerate}
\item If $i=1$, return $E_{x_1}$
\item Else return $\FEDec_{E_2}\Big(\big(\eval(i-1,E_1,x_{[i-1]})\big)_{x_i}\Big)$
\end{enumerate}
\end{varwidth}}
\end{figure}

\label{sec:iO-FE}

\bibliography{refs}{}
\bibliographystyle{alpha}

\end{document}
